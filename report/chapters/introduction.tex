\chapter{Introdução}
Neste trabalho foi implementado um método computacional de maneira a resolver
a equação de advecção de forma numérica.

Para melhor entender o desenvolvimento, é necessária introdução dos\linebreak
conceitos-chave utilizados, alguns dos quais já foram apresentados no primeiro
trabalho.

\section{A Equação de Advecção}
A equação de advecção é obtida a partir da equação de advecção-difusão,
introduzida no primeiro trabalho como exemplo de modelagem do escoamento de um
contaminante em um córrego. A parte advectiva desta trata apenas do
carregamento da substância devido a velocidade da correnteza. A forma mais
geral da equação de advecção é
\begin{equation}
    \frac{\partial c}{\partial t} + \frac{\partial}{\partial x}(uc) = 0
\end{equation}
onde $c$ indica a concentração e $u$ a velocidade. Para este trabalho
assume-se um $u$ constante e maior que zero, denotado como $\bar{u}$. Sendo
assim, a forma final equação da advecção a ser utilizada neste trabalho é
\begin{equation}\label{adv}
    \frac{\partial c}{\partial t} + \bar{u}\frac{\partial c}{\partial x} = 0
\end{equation}

\section{Método dos Volumes Finitos}
O método dos volumes finitos tem como finalidade a discretização do domínio
espacial. Este é subdividido em um conjunto de volumes finitos e as variáveis
dependentes são determinadas como médias volumétricas sobre estes volumes,
avaliadas nos centros dos mesmos. Neste trabalho, serão utilizados quatro
métodos numéricos --- Forward Time-Backward Space (FTBS), Lax-Friedrichs,
Lax-Wendroff e Beam-Warming --- visando resolver a equação da advecção.