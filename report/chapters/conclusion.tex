\chapter{Conclusão}
<modificar conclusão>

Equações diferenciais modelam quase tudo a nossa volta. A imagem introdutória,
do despejo de esgoto, é exemplo disso. Ser capaz de aliar a análise matemática
ao poder de processamento de um computador desbloqueia inúmeras possibilidades,
principalmente no campo de engenharia.

Muitas vezes, equações diferenciais parciais não possuem solução analítica.
Apesar de isto parecer um obstáculo intransponível, é possível resolvê-las
usando alguns métodos alternativos.

Com o auxílio do Método dos Volumes Finitos, é viável resolver EDPs muito
difíceis desde que se sua forma discretizada respeite as regras de
consistência, convergência e estabilidade. Assim será possível analisar seu
comportamento ao longo do tempo, e determinar o seu significado físico.

Neste trabalho foi possível perceber que cada parâmetro tem uma influência
sobre o comportamento da simulação da EDP. Em alguns casos, pequenas alterações
são insignificantes e, em outros, extremamente notáveis. O conhecimento do
comportamento destes parâmetros aplicado a ferramenta computacional agiliza o
desenvolvimento de projetos de engenharia, e viabiliza alguns que seriam
impraticáveis de serem resolvidos manualmente.