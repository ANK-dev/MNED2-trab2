\chapter{Introdução}
Neste trabalho foi implementado um método computacional de maneira a resolver
a equação de advecção de forma numérica.

Para melhor entender o desenvolvimento, é necessária introdução dos\linebreak
conceitos-chave utilizados, alguns dos quais já foram apresentados no primeiro
trabalho.

\section{A Equação de Advecção}
A equação de advecção é obtida a partir da equação de advecção-difusão,
introduzida no primeiro trabalho como exemplo de modelagem do escoamento de um
contaminante em um córrego. A parte advectiva desta trata apenas do
carregamento da substância devido a velocidade da correnteza. A forma mais
geral da equação de advecção é
\begin{equation}
    \frac{\partial c}{\partial t} + \frac{\partial}{\partial x}(uc) = 0
\end{equation}
onde $c$ indica a concentração e $u$ a velocidade. Para este trabalho
assume-se um $u$ constante e maior que zero, denotado como $\bar{u}$. Sendo
assim, a forma final equação da advecção a ser utilizada neste trabalho é
\begin{equation}\label{adv}
    \frac{\partial c}{\partial t} + \bar{u}\frac{\partial c}{\partial x} = 0
\end{equation}

\section{Método dos Volumes Finitos}
O método dos volumes finitos tem como finalidade a discretização do domínio
espacial. Este é subdividido em um conjunto de volumes finitos e as variáveis
dependentes são determinadas como médias volumétricas sobre estes volumes,
avaliadas nos centros dos mesmos. Partindo de um problema unidimensional, temos
um caso particular da lei de conservação discretizada,
\begin{equation}\label{lei de cons.}
    Q_i^{n+1} = Q_i^n - \frac{\Delta t}{\Delta x}\left(
        F_{i+1/2}^n - F_{i-1/2}^n
    \right)
\end{equation}
onde $F$ indica o fluxo, definido como,
\[
    F_{i \pm 1/2}^n
    \approx
    \frac{1}{\Delta t} \int_{t_n}^{t_{n+1}} f(x_{i \pm 1/2},t)dt
\]
e $Q$ indica as concentrações na malha, definido como
\[
    Q_i^n \approx \frac{1}{\Delta x} \int_{x_{i-1/2}}^{i_{i+1/2}} \phi(x,t_n)dx
\]

Considerando que, para problemas hiperbólicos, a velocidade de propagação é
finita, é possível definir uma representação para os fluxos nas faces do volume
de controle em função de $Q^n$, isto é,
\[
    F_{i-1/2}^n = \mathcal{F}(Q_{i-1}^n, Q_i^n)
\]
\[
    F_{i+1/2}^n = \mathcal{F}(Q_i^n, Q_{i+1}^n)
\]
Reescrevendo a Eq.\ \ref{lei de cons.} em função de $\mathcal{F}$ obtém-se,
\begin{equation}
    Q_i^{n+1} = Q_i^n - \frac{\Delta t}{\Delta x}\bigg[
        \mathcal{F}(Q_i^n,Q_{i+1}^n) - \mathcal{F}(Q_{i-1}^n,Q_i^n)
    \bigg]
\end{equation}
de forma que o valor futuro $Q_i^{n+1}$ depende explicitamente dos valores
anteriores $Q^n$ na vizinhança do volume $i$ e das funções $\mathcal{F}$.

Neste trabalho, serão utilizados quatro métodos numéricos baseados no método
dos volumes finitos --- Forward Time-Backward Space (FTBS),\linebreak
Lax-Friedrichs, Lax-Wendroff e Beam-Warming --- visando resolver a equação da
advecção.