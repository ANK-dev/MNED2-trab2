\chapter{Metodologia}
Neste capítulo serão abordados os passos e métodos utilizados para se
obter a solução numérica do problema proposto.

\section{Condições Inicial e de Contorno}
A resolução de qualquer equação diferencial parcial (EDP) requer a
determinação de sua condição(ões) inicial(ais) e de contorno. Como
proposto pelo trabalho, a concentração inicial da malha é dada pela
seguinte equação
\begin{equation}
    c(x,0) = e^{-A(x - B)} + s(x)
\end{equation}
e as condições de contorno são dadas pelas seguintes equações

% Equações lado-a-lado
\medskip
\noindent
\begin{minipage}{.5\linewidth}
    \begin{equation}
        \left( \frac{\partial c}{\partial x} \right)_{x=0}^t = 0
    \end{equation}
\end{minipage}%
\begin{minipage}{.5\linewidth}
    \begin{equation}
        \left( \frac{\partial c}{\partial x} \right)_{x=L_x}^t = 0
    \end{equation}
\end{minipage}

\medskip
\noindent para o contorno esquerdo e direito, respectivamente.

Usando o conceitos de volumes fantasmas, é possível determinar o valor dos
volumes no contorno, através das seguintes aproximações, para o contorno
esquerdo,

\[
    \left(\frac{\partial c}{\partial c}\right)_{x=0}^t
    \approx
    \frac{Q_1^n - Q_0^n}{\Delta x} = 0
\]
\begin{equation}
    \therefore\ Q_1^n = Q_0^n
\end{equation}
e para o contorno direito,
\[
    \left(\frac{\partial c}{\partial c}\right)_{x=L_x}^t
    \approx
    \frac{Q_{nx+1}^n - Q_{nx}^n}{\Delta x} = 0
\]
\begin{equation}
    \therefore\ Q_{nx+1}^n = Q_{nx}^n
\end{equation}

\section{Métodos Numéricos}
Os métodos numéricos utilizados para a resolução da equação de advecção são
diversos. Nesta seção serão tratados os quatro métodos utilizados neste
trabalho.

\subsection{Métodos \textit{Upwind}}
Problemas hiperbólicos, como a equação da advecção, possuem informação (ondas)
que se propagam com uma velocidade e sentido característico. A utilização de
métodos \textit{upwind} leva em conta essa característica, permitindo uma
modelagem mais acurada do fenômeno tratado. O método \textit{upwind} escolhido
para a resolução da equação da advecção é o \textit{forward time-backward space}
(FTBS).

\subsubsection{Forward Time-Backward Space (FTBS)}
O FTBS trata da ideia de que, para a equação da advecção unidimensional, há
apenas uma única onda que se propaga. O método \textit{upwind} determina o
valor de $Q_i^{n+1}$, onde, para um $\bar{u} > 0$, resulta em um fluxo da
esquerda para a direita, de forma que a concentração de cada volume $Q_i^{n+1}$
depende dos volumes atual $Q_i^n$ e anterior $Q_{i-1}^n$. Sua discretização é:
\begin{equation}\label{FTBS}
    Q_i^{n+1} = Q_i^n - \frac{\bar{u}\Delta t}{\Delta x} \left(
        Q_i^n - Q_{i-1}^n
    \right)
\end{equation}
aplicando as condições de contorno, obtém-se,
\[
    Q_i^{n+1} = Q_i^n - \frac{\bar{u}\Delta t}{\Delta x} \left(
        Q_1^n - Q_1^n
    \right)
\]
\begin{equation}
    \therefore\ Q_1^{n+1} = Q_1^n
\end{equation}
para o primeiro elemento da malha (contorno esquerdo). O último elemento da
malha (contorno direito) no método FTBS não necessita do volume fantasma, pois
depende apenas do elemento atual $Q_i^n$ e do anterior $Q_{i-1}^n$, portanto, a
Eq.\ \ref{FTBS} aplica-se ao contorno direito.

\subsection{Lax-Friedrichs (L-F)}
Para Lax-Friedrichs (L-F), tem-se a seguinte discretização
\begin{equation}\label{L-F}
    Q_i^{n+1} = \frac{Q_{i+1}^n + Q_{i-1}^n}{2}
    - \frac{\bar{u}\Delta t}{2\Delta x}\left(
    Q_{i+1}^n - Q_{i-1}^n
    \right)
\end{equation}
aplicando as condições de contorno, obtém-se,
\begin{equation}
    Q_1^{n+1} = \frac{Q_2^n + Q_1^n}{2}
    - \frac{\bar{u}\Delta t}{2\Delta x}\left(
    Q_2^n - Q_1^n
    \right)
\end{equation}
para o primeiro elemento da malha (contorno esquerdo). Enquanto para o último
elemento da malha (contorno direito), obtém-se,
\begin{equation}
    Q_{nx}^{n+1} = \frac{Q_{nx}^n + Q_{nx-1}^n}{2}
    - \frac{\bar{u}\Delta t}{2\Delta x}\left(
    Q_{nx}^n - Q_{nx-1}^n
    \right)
\end{equation}

\subsection{Métodos de Alta Resolução}
A utilização de métodos \textit{upwind} de primeira ordem, apesar de simples,
acaba por introduzir significativa difusão numérica, impactando negativamente a
acurácia da solução. Os métodos de alta resolução introduzem um termo
corretivo, de maneira a minimizar a influência da difusão numérica sobre o
resultado final. Os métodos de alta resolução escolhidos para a resolução da
equação da advecção são o Lax-Wendroff (L-W) e Beam-Warming (B-W).

\subsubsection{Lax-Wendroff (L-W)}
Para Lax-Wendroff (L-W), tem-se a seguinte discretização
\begin{equation}\label{L-W}
    Q_i^{n+1} = Q_i^n - \frac{\bar{u}\Delta t}{2\Delta x}\left(
        Q_{i+1}^n - Q_{i-1}^n
    \right) + \frac{\bar{u}^2{\Delta t}^2}{2{\Delta x}^2}\left(
        Q_{i+1}^n - 2Q_i^n + Q_{i-1}^n
    \right)
\end{equation}
aplicando as condições de contorno, obtém-se,
\[
    Q_1^{n+1} = Q_1^n - \frac{\bar{u}\Delta t}{2\Delta x}\left(
        Q_2^n - Q_1^n
    \right) + \frac{\bar{u}^2{\Delta t}^2}{2{\Delta x}^2}\left(
        Q_2^n - 2Q_1^n + Q_1^n
    \right)
\]
\begin{equation}\label{L-W esq}
    Q_1^{n+1} = Q_1^n - \frac{\bar{u}\Delta t}{2\Delta x}\left(
        Q_2^n - Q_1^n
    \right) + \frac{\bar{u}^2{\Delta t}^2}{2{\Delta x}^2}\left(
        Q_2^n - Q_1^n
    \right)
\end{equation}
para o primeiro elemento da malha (contorno esquerdo). Enquanto para o último
elemento da malha (contorno direito), obtém-se,
\[
    Q_{nx}^{n+1} = Q_{nx}^n - \frac{\bar{u}\Delta t}{2\Delta x}\left(
        Q_{nx}^n - Q_{nx-1}^n
    \right) + \frac{\bar{u}^2{\Delta t}^2}{2{\Delta x}^2}\left(
        Q_{nx}^n - 2Q_{nx}^n + Q_{nx-1}^n
    \right)
\]
\begin{equation}
    Q_{nx}^{n+1} = Q_{nx}^n - \frac{\bar{u}\Delta t}{2\Delta x}\left(
        Q_{nx}^n - Q_{nx-1}^n
    \right) + \frac{\bar{u}^2{\Delta t}^2}{2{\Delta x}^2}\left(
        Q_{nx-1}^n - Q_{nx}^n
    \right)
\end{equation}

\subsubsection{Beam-Warming (B-W)}
Para Beam-Warming (B-W), tem-se a seguinte discretização
\begin{equation}\label{B-W}
    % Redimensionamento para largura da página `\resizebox'
    \resizebox{0.8\textwidth}{!}
    {$
    Q_i^{n+1} = Q_i^n - \frac{\bar{u}\Delta t}{2\Delta x}\left(
        3Q_i^n - 4Q_{i-1}^n + Q_{i-2}^n
    \right) + \frac{\bar{u}^2{\Delta t}^2}{2{\Delta x}^2}\left(
        Q_i^n - 2Q_{i-1}^n + Q_{i-2}^n
    \right)
    $}
\end{equation}
onde, para o primeiro elemento (contorno esquerdo), devido ao termo
$Q_{i-2}^n$, não é possível aplicar as condições de contorno diretamente. Sendo
assim, apenas para o primeiro elemento da malha, será utilizado o método L-W.
Desta forma, para $i=1$ a discretização utilizada é a Eq.\ \ref{L-W esq}. Para
o segundo elemento, aplica-se a condição de contorno esquerda, resultando na
seguinte discretização,
\[
    % Redimensionamento para largura da página `\resizebox'
    \resizebox{0.8\textwidth}{!}
    {$
    Q_2^{n+1} = Q_2^n - \frac{\bar{u}\Delta t}{2\Delta x}\left(
        3Q_2^n - 4Q_1^n + Q_1^n
    \right) + \frac{\bar{u}^2{\Delta t}^2}{2{\Delta x}^2}\left(
        Q_2^n - 2Q_1^n + Q_1^n
    \right)
    $}
\]
\begin{equation}
    % Redimensionamento para largura da página `\resizebox'
    \resizebox{0.8\textwidth}{!}
    {$
    Q_2^{n+1} = Q_2^n - \frac{3\bar{u}\Delta t}{2\Delta x}\left(
        Q_2^n - Q_1^n
    \right) + \frac{\bar{u}^2{\Delta t}^2}{2{\Delta x}^2}\left(
        Q_2^n - Q_1^n
    \right)
    $}
\end{equation}
Assim como no método FTBS, o último elemento da malha (contorno direito) do
método B-W não necessita do volume fantasma, pois depende apenas do elemento
atual $Q_i^n$ e dos anteriores $Q_{i-1}^n$ e $Q_{i-2}^n$, portanto, a Eq.\
\ref{B-W} aplica-se ao contorno direito.

\section{Estabilidade}
Se trata do comportamento do algoritmo e seus valores numéricos frente aos
parâmetros de entrada. Um algoritmo estável se comporta de maneira esperada
frente a uma faixa específica de valores de entrada.

A partir da condição de estabilidade para a equação da advecção-difusão,
utilizada no primeiro trabalho,
\begin{equation}
    \Delta t
    \leq
    \frac{1}{\frac{2\alpha}{{\Delta x}^2} + \frac{\bar{u}}{\Delta x}}
\end{equation}
considerando que não há componente difusivo $\alpha$ na equação de advecção, o
mesmo pode ser igualado a 0.
\[
    \Delta t
    \leq
    \frac{1}{\frac{2(0)}{{\Delta x}^2} + \frac{\bar{u}}{\Delta x}}
\]
\[
    \Delta t \leq \frac{1}{\frac{\bar{u}}{\Delta x}}
\]
\begin{equation}
    \therefore\ \Delta t \leq \frac{\Delta x}{\bar{u}}
\end{equation}
obtém-se assim a condição de estabilidade para os métodos numéricos para a
equação da advecção. Trata-se de um método \emph{condicionalmente estável}, ou
seja que depende de uma faixa de valores para garantir a estabilidade de seu
funcionamento.

\section{Programação}
Aliado destes conceitos, foi possível construir um programa em linguagem C que
calcula as concentrações para cada célula ao longo do tempo, para cada método
aqui citado, exportando arquivos de texto com os resultados; estes arquivos,
então, são lidos por um \textit{script} Python que gera os gráficos
correspondentes.

O programa principal possui a seguinte estrutura, descrita em C:

\begin{Verbatim}[fontsize=\footnotesize]
// Vetores para concentração no tempo `n' e no tempo `n+1', respectivamente
double Q_old[];
double Q_new[];

// Para cada método, os vetores são inicializados e as concentrações são
// calculadas

// Método FTBS
// Inicializa ambos os vetores com a função de concentração inicial
initializeArray(Q_old, nx, A, B, C, D, E)
initializeArray(Q_new, nx, A, B, C, D, E)

// Calcula Q através do método FTBS
calculateQ_FTBS(Q_old, Q_new);

// Imprime na tela e salva os resultados no arquivo de texto
printAndSaveResults(Q_new, nx, FTBS);

// Método L-F
initializeArray(Q_old, nx, A, B, C, D, E)
initializeArray(Q_new, nx, A, B, C, D, E)
calculateQ_LF(Q_old, Q_new);
printAndSaveResults(Q_new, nx, LF);

// Método L-W
initializeArray(Q_old, nx, A, B, C, D, E)
initializeArray(Q_new, nx, A, B, C, D, E)
calculateQ_LW(Q_old, Q_new);
printAndSaveResults(Q_new, nx, LW);

// Método B-W
initializeArray(Q_old, nx, A, B, C, D, E)
initializeArray(Q_new, nx, A, B, C, D, E)
calculateQ_BW(Q_old, Q_new);
printAndSaveResults(Q_new, nx, BW);
\end{Verbatim}

\noindent A função \verb|initializeArray| possui a seguinte estrutura:
\begin{Verbatim}[fontsize=\footnotesize]
int i;
double x, s;

// Laço `for' percorre todos os índices do vetor
for (i = 0; i < nx; ++i) {

    // Posição `x' (m) é definida como índice do volume * largura do volume
    x = i * Delta_x;

    // `s' recebe o valor de `E' somente se C <= x <= D
    s = (x >= C && x <= D ? E : 0);

    // Cada índice do vetor `Q' é inicializado segundo a condição inicial
    Q[i] = exp( -A * ((x - b)*(x - b)) ) + s;
}
\end{Verbatim}

\noindent Cada função \verb|calculateQ| possui a seguinte estrutura:

\begin{Verbatim}[fontsize=\footnotesize]
// Cálculo de Q, iterado ao longo do tempo
do {

    // Cálculo do volume da fronteira esquerda, onde `leftBoundary' é a função
    // da fronteira esquerda específica a cada método
    Q_new[0] = leftBoundary(method, 0);

    // Cálculo de volumes do centro da malha, onde `center' é a função do centro
    // da malha específica a cada método. Este laço começa em `i = 2' caso o
    // método seja B-W.
    for (i = (method == BW ? 2 : 1); i < nx - 1; ++i) {
        Q_new[i] = center(method, i);
    }

    // Cálculo do volume da fronteira direita, onde `rightBoundary' é a função
    // da fronteira direita específica a cada método
    Q_new[i] = rightBoundary(method, i);

    // Vetor `Q_old' é atualizado com valores do `Q_new' para a próxima iteração
    for (i = 0; i < nx; ++i) {
        Q_old[i] = Q_new[i];
    }

// Incrementa passo de tempo
} while ( (t += Delta_t) <= t_final);
\end{Verbatim}

São definidos dois vetores, \verb|Q_old[]| e \verb|Q_new[]|, que correspondem as
concentrações $Q$ no tempo $n$ e $n+1$, respectivamente. Antes do cálculo das
concentrações, os vetores são inicializados, em um simples laço \verb|for|,
seguindo a função de concentração inicial $e^{-A(x-b)^2} + s(x)$.

A cada iteração do laço \verb|do-while|, o tempo \verb|t| é incrementado por uma
quantidade \verb|Delta_t|, que obedece as regras de estabilidade descritas na
seção anterior. Ao longo da iteração, o vetor \verb|Q_new[]| é calculado para as
fronteiras e para o centro da malha, em função de \verb|Q_old[]|. Antes do fim
da iteração, os vetores \verb|Q_old[]| são atualizados com os valores de
\verb|Q_new[]|, o tempo é incrementado, e então a nova iteração é iniciada.

Ao fim da execução, o vetor \verb|Q_new[]|, terá os resultados da concentração
de cada volume da malha, correspondente a cada índice do vetor, no tempo
\verb|t = t_final|. Os pares índice-concentração são exportados em um arquivo
de texto, para cada método, linha-a-linha, para serem lidos e plotados pelo
\textit{script} Python.
